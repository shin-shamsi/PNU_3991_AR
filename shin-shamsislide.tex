 % !TeX TS-program = xelatex

\documentclass{beamer}
%Set the slide theme
%Change to meet your taste
% Madrid, Copenhagen, Berlin, ... works
\usetheme{Madrid} 
%\usetheme{metropolis}


\usepackage{xecolor}
\usepackage{amsmath}
%\usefonttheme[onlymath]{serif} %Change the math font

\usepackage{xepersian}
\settextfont{XB Roya}

%---------------------------------------------------------------------------------
% Seetings to force Beamer works with Xepersian and RTL typesetting
%-------------------------------------------------------------------------------
%\raggedleft

% For right to left lists (itemize and enumerate)
\makeatletter
\newcommand{ \RTList}{\raggedleft\rightskip\@totalleftmargin} 
\makeatother
% Correct the bullet for RTL texts
\setbeamertemplate{itemize item}{\scriptsize\raise1.25pt%
 \hbox{\donotcoloroutermaths$\blacktriangleleft$}} 

% To force beamer use numbering in captions
\setbeamertemplate{caption}[numbered]{}% Number float-like environments



%---------------------------------------------------------------------------------
\title{
تهیه اسلاید ترجمه صفحات 103تا105 کتاب \\Strategies Methodes e-searche
}

\subtitle{ساخت اسلاید بیمربه روش ساده‌}
\author{شهریارشمسی }
\institute{دانشگاه پیام نور تهران شمال}
\date{پائیز1399} 


\begin{document}
\begin{persian}
%------------------------------------------
% Title page
%------------------------------------------
\begin{frame}
\maketitle
\end{frame}

% To adjust the paragraphs in RTL
\everypar{\rightskip\rightmargin}
%-------------------------------------------------------------------------------
\begin{frame}{سرآغاز}
\section{مبانی}
\subsection{متن ساده}
دیگران . در حالی که مصاحبه گروه کانونی در اصل به عنوان راهی برای دستیابی به اولویت‌های محصولات مصرفی بود , این تکنیک در نهایت به روش تحقیق علوم اجتماعی و علوم اجتماعی راه یافت . نشان‌داده شده‌است که ظرفیت جذب داده‌های کیفی غنی و معتبر را دارد . علاوه بر این , طبق نظر پاتن , مصاحبه گروه کانونی , کنترل‌های کیفی را بر روی جمع‌آوری داده‌ها ارائه می‌کند , به طوری که برخی از آن‌ها تمایل به پرسش و حذف دیدگاه‌های غلط یا افراطی دارند . نتیجه , تمایل به تمرکز بر مهم‌ترین موضوعات و مسائل است و تا حدی که یک ارتباط منسجم در میان شرکت کنندگان وجود دارد , در میان شرکت کنندگان وجود دارد.
این روش مصاحبه گروه کانونی نامیده می‌شود زیرا به دو روش متمرکز شده‌است . اول اینکه , شرکت کنندگان این گروه به نوعی شبیه به هم هستند . دوم , هدف جمع‌آوری داده‌ها در مورد یک موضوع منفرد ( یا محدوده محدود موضوعات ) است . آن‌ها اغلب توسط پرسش‌های مطرح‌شده توسط محقق و با تاکید بر کسب بینش از طریق نظرات گروهی به جای حقایق خاص هدایت می‌شوند . این فرمت راهی مناسب برای جمع‌آوری دانش فردی اعضا و الهام بخشیدن به نگرش‌ها و راه‌حل‌هایی است که دستیابی به روش‌های دیگر مصاحبه دشوار است 
ترکیب رسانه مورد استفاده در فرآیند گروه متمرکز , تعمیم ویژگی‌های همه گروه‌های متمرکز را دشوار می‌سازد . با این حال , آنچه که منحصرا ً یک گروه متمرکز را از یک مصاحبه متمایز می‌کند , ظرفیت شرکت کنندگان برای اشتراک و ایجاد نظرات و نگرانی‌های دیگر شرکت کنندگان است . قبل از اینترنت , چیزی به عنوان یک گروه تمرکز غیر همزمان وجود نداشت ; هر چند که می‌توان از گروه‌های تمرکز محور استفاده کرد , محققان به طور فعال از این تکنیک استفاده نکردند . فرم‌های غالب ارتباطات ناهمزمان در اینترنت شامل ایمیل و کنفرانس کامپیوتری بوده‌است . در حال حاضر ما شاهد انتخاب سریع تعداد بیشتری از رسانه‌های ارتباط غیر همزمان برای هدایت گروه‌های تمرکز مبتنی بر شبکه هستیم. در طرف همزمان ، چت کردن متنی رایج‌ترین و قابل‌دسترس ترین راه برای هدایت گروه‌های تمرکز بلادرنگ است . این حالت تعامل از نرم‌افزاری مانند Iای سی کیو نت متینگ ، یا یکی از برنامه‌های نرم‌افزاری چت وب که بر پایه جاوا قرار دارند استفاده می‌کند تا نظرات شرکت کنندگان را به گونه‌ای که تایپ می‌کنند ، به اشتراک بگذارند . متن را می‌توان با مشاهده اشیا ، به اشتراک گذاری برنامه‌ها ، یا به اشتراک گذاری یک فضای مشترک از طریق محیط‌های مجازی مبتنی بر متن ( وی آر ) و یا از طریق محیط‌های وی آر دو بعدی مانند کاخ ( مثلا ، www.thepalace.com ) یا جهان‌های مجازی ( به عنوان مثال ، www.worlds.net ) ، افزایش داد . در نهایت ، گروه‌های تمرکز می‌توانند با استفاده از کنفرانس صوتی یا تصویری انجام شوند . تا همین اواخر ، سخت‌افزار مورد نیاز ، سخت‌افزار کاربر نهایی و پهنای باند مانع استفاده از این اشکال غنی‌تر و طبیعی‌تر از ارتباطات بر روی شبکه شده‌است . با این حال ، توسعه سیستم‌های صوتی و ویدئو کنفرانس ( مراجعه به پنجره‌های www.microsoft.com / / / و (www.centra.com netmeeting ) و در دسترس بودن ارتباط اینترنتی با سرعت بالا در خانه و در محیط کار وعده افزایش استفاده از گروه‌های تمرکز مبتنی بر رسانه را افزایش می‌دهد. . 
در حال حاضر , بیشتر گروه‌های تمرکز محور با استفاده از نرم‌افزارهای مبتنی بر متن و یا نرم‌افزارهای همزمان انجام می‌شوند . بر این اساس , این فصل بر گروه‌های متمرکز متن محور متمرکز است . همانطور که خدمات باند پهن تکثیر می‌شوند و به طور گسترده قابل‌استفاده و مقرون‌به‌صرفه خواهند شد , ما احتمالا ً شاهد افزایش در ویدئو و / یا گروه‌های کانونی غیر همزمان و غیر همزمان خواهیم بود . همانطور که این خدمات چندرسانه‌ای گروه‌های تمرکز بر پایه خالص , بیشتر شبیه گروه‌های متمرکز با چهره هستند . گروه‌های متمرکز با تمرکز رو به رو به طور فزاینده‌ای مرتبط و مفید خواهند شد 

\end{frame}

%-------------------------------------------------------------------------------
\section{سرآغاز}
%-------------------------------------------------------------------------------
\begin{frame}{ادامه سرآغاز}

گروهی از گروه‌های متمرکز این است که به پاسخ دهندگان اجازه می‌دهند واکنشی نشان دهند و پاسخ دهند . نتیجه می‌تواند تاثیر هم افزایی و پویا بر رفتار گروهی باشد , که اغلب منجر به داده یا ایده‌هایی می‌شود که ممکن است در مصاحبه‌های فردی جمع‌آوری نشوند . علاوه بر این, از آنجا که گروه‌های تمرکز تمایل به ارائه چک و توازن بین اعضای گروه دارند تا دیدگاه‌های غلط یا افراطی را حذف کنند , برای محقق آسان است که میزان دیدگاه‌های سازگار و اشتراکی را ارزیابی کند . با توجه به این مزایا , مصاحبه با گروهی از افراد در یک موضوع متمرکز می‌تواند راه قدرتمندی برای جمع‌آوری داده‌ها باشد با این حال , باید تاکید کرد که گروه‌های تمرکز بازخورد را از روش تصادفی انتخاب‌شده نشان نمی‌دهند بلکه از افراد عمدا ً انتخاب شده‌اند . به این ترتیب  نتایج حاصل از مصاحبه‌های گروه متمرکز نباید به جمعیت‌های بزرگ‌تر تعمیم داده شوند.
 در نهایت , گروه‌های تمرکز بر پایه , به دلیل نیاز به مشارکت افراد از چندین منطقه جغرافیایی مختلف , بر روی گروه‌های متمرکز انتخاب می‌شوند . زمان و هزینه سفر افراد پراکنده , اغلب بازدارنده محسوب می‌شود . شبکه یک محیط را فراهم می‌کند که به موجب آن محقق می‌تواند یک گروه متمرکز هزینه - موثر را اجرا کند 
\end{frame}

%-------------------------------------------------------------------------------
\begin{frame}{انواع مختلف گروه‌های تمرکز محور}
گروه‌های تمرکز مبتنی بر شبکه را می‌توان در اینترنت یا بصورت همزمان یا با نرم‌افزار متن محور و / یا نرم‌افزارهای صوتی و تصویری انجام داد . جدول ۲ - ۱ مثال‌هایی از انواع مختلف گروه‌های متمرکز مبتنی بر شبکه را ارائه می‌کند که توسط تمایزهای بدون متن و مبتنی بر متن طبقه‌بندی شده‌اند . همانطور که این جدول نشان می‌دهد, چهار گروه تمرکز محور مبتنی بر متن وجود دارند : همگام و مبتنی بر متن ; همگام و مبتنی بر متن ; ناهمگام و مبتنی بر متن

\end{frame}


%-------------------------------------------------------------------------------
\begin{frame}{انواع مختلف گروه‌های تمرکز محور}
	ترکیب رسانه مورد استفاده در فرآیند گروه متمرکز , تعمیم ویژگی‌های همه گروه‌های متمرکز را دشوار می‌سازد . با این حال , آنچه که منحصرا ً یک گروه متمرکز را از یک مصاحبه متمایز می‌کند , ظرفیت شرکت کنندگان برای اشتراک و ایجاد نظرات و نگرانی‌های دیگر شرکت کنندگان است . قبل از اینترنت , چیزی به عنوان یک گروه تمرکز غیر همزمان وجود نداشت ; هر چند که می‌توان از گروه‌های تمرکز محور استفاده کرد , محققان به طور فعال از این تکنیک استفاده نکردند . فرم‌های غالب ارتباطات ناهمزمان در اینترنت شامل ایمیل و کنفرانس کامپیوتری بوده‌است .در حال حاضر ما شاهد انتخاب سریع تعداد بیشتری از رسانه‌های ارتباط غیر همزمان برای هدایت گروه‌های تمرکز مبتنی بر شبکه هستیم. در طرف همزمان ، چت کردن متنی رایج‌ترین و قابل‌دسترس ترین راه برای هدایت گروه‌های تمرکز بلادرنگ است . این حالت تعامل از نرم‌افزاری مانند Iای سی کیو نت متینگ ، یا یکی از برنامه‌های نرم‌افزاری چت وب که بر پایه جاوا قرار دارند استفاده می‌کند تا نظرات شرکت کنندگان را به گونه‌ای که تایپ می‌کنند ، به اشتراک بگذارند .
\end{frame}	
\begin{frame}{انواع مختلف گروه‌های تمرکز محور}
	متن را می‌توان با مشاهده اشیا ، به اشتراک گذاری برنامه‌ها ، یا به اشتراک گذاری یک فضای مشترک از طریق محیط‌های مجازی مبتنی بر متن ( وی آر ) و یا از طریق محیط‌های وی آر دو بعدی مانند کاخ ( مثلا ، www.thepalace.com ) یا جهان‌های مجازی ( به عنوان مثال ، www.worlds.net ) ، افزایش داد . در نهایت ، گروه‌های تمرکز می‌توانند با استفاده از کنفرانس صوتی یا تصویری انجام شوند . تا همین اواخر ، سخت‌افزار مورد نیاز ، سخت‌افزار کاربر نهایی و پهنای باند مانع استفاده از این اشکال غنی‌تر و طبیعی‌تر از ارتباطات بر روی شبکه شده‌است . با این حال ، توسعه سیستم‌های صوتی و ویدئو کنفرانس ( مراجعه به پنجره‌های www.microsoft.com / / / و (www.centra.com netmeeting ) و در دسترس بودن ارتباط اینترنتی با سرعت بالا در خانه و در محیط کار وعده افزایش استفاده از گروه‌های تمرکز مبتنی بر رسانه را افزایش می‌دهد. 


\end{frame}


%-------------------------------------------------------------------------------
\begin{frame}{انواع مختلف گروه‌های تمرکز محور}

در حال حاضر , بیشتر گروه‌های تمرکز محور با استفاده از نرم‌افزارهای مبتنی بر متن و یا نرم‌افزارهای همزمان انجام می‌شوند . بر این اساس , این فصل بر گروه‌های متمرکز متن محور متمرکز است . همانطور که خدمات باند پهن تکثیر می‌شوند و به طور گسترده قابل‌استفاده و مقرون‌به‌صرفه خواهند شد , ما احتمالا ً شاهد افزایش در ویدئو و / یا گروه‌های کانونی غیر همزمان و غیر همزمان خواهیم بود . همانطور که این خدمات چندرسانه‌ای گروه‌های تمرکز بر پایه خالص , بیشتر شبیه گروه‌های متمرکز با چهره هستند . گروه‌های متمرکز با تمرکز رو به رو به طور فزاینده‌ای مرتبط و مفید خواهند شد


\end{frame}

----------
\begin{frame}{مزایا و معایب گروه های متمرکز چهره به چهره }
تنها اخیرا ً محققان قادر به استفاده از اینترنت برای هدایت گروه‌های تمرکز وابسته به آموزشی بوده‌اند . گروه‌های تمرکز مبتنی برشبکه هم سرعت و هم هزینه را کاهش می‌دهند ( وان نیس 1999). در حالی که گروه‌های تمرکز مبتنی بر نت در مرحله توسعه و توسعه متمرکز هستند ، به نظر می‌رسد که به ویژه در حذف موانعی که بسیاری از محققان هنگام اجرای گروه‌های تمرکز رو به رو تجربه می‌کنند ، موثر هستند . به طور خاص ، آن‌ها می‌توانند مشارکت و موانع هزینه را کاهش داده یا حذف کنند . به عنوان مثال ، اگر محقق الکترونیکی و شرکت کنندگان از لحاظ جغرافیایی پراکنده شوند ، گروه‌های تمرکز مبتنی بر نت به آن‌ها اجازه می‌دهد تا از خانه و / یا دفاتر خود شرکت کنند ، در نتیجه هزینه‌های سفر حذف می‌شوند . تحلیل هزینه وان " نشان می‌دهد که علاوه بر صرفه‌جویی در سفر ، حدود ۲۰ درصد صرفه‌جویی در هزینه در اجرای گروه تمرکز در مقایسه با گروه‌های تمرکز رو به صورت وجود دارد . به عنوان مثال ، چنین هزینه‌هایی بعنوان غذا ، نوشیدنی و اجاره اتاق نباید در یک گروه تمرکز مبتنی بر نت ایجاد شود . علاوه بر این مزایا ، بحث‌های آنلاین می‌توانند به طور خودکار آرشیو شوند ، فرآیند رونویسی و خطای تفسیر رونویسی را حذف کنند. 
\end{frame}

\begin{frame}{مزایا و معایب گروه های متمرکز چهره به چهره }
	در نهایت , گروه‌های تمرکز محور ممکن است کشمکش‌های قدرت را کاهش دهند که اغلب در گروه‌های تمرکز رو به رو در نتیجه نظرات متعارض در زمانی رخ می‌دهد که تفاوت وضعیت در میان شرکت کنندگان وجود دارد با این حال , لازم نیست شرکت کنندگان گروه تمرکز محور هویت واقعی خود را به سایر اعضای گروه نشان دهند . با توجه به این توانایی برای ایجاد یک اسم مستعار برای هر شرکت‌کننده ( بسته به رسانه ) و اینکه گروه‌های تمرکز محور توانایی پیوستن به داده‌های جغرافیایی را دارند ( بنابراین کاهش احتمال مشارکت کنندگانی که یکدیگر را می‌شناسند ) , مشکلات قدرت و مسایل محرمانگی را می ‌توان کاهش داد. 
\end{frame}

\begin{frame}{مزایا و معایب گروه های متمرکز چهره به چهره }
با توجه به این مزایا , اکتشافات اولیه با توجه به کیفیت داده‌های جمع‌آوری‌شده , با نتایج مختلفی مواجه شده‌است . برای مثال , مشاهده می‌شود که یک اشکال در گروه‌های متمرکز بر مبنای متن , این است که اغلب عمق کمتری در پاسخ‌های شرکت‌کنندگان و نیز از دست دادن اشارات غیر کلامی وجود دارد . علایم عصبی - tic , به ویژه , منبع بسیار ارزشمندی از داده‌ها در گروه‌های تمرکز رو به رو هستند - علاوه بر آنچه گفته می‌شود . علاوه بر این, فان تین اشاره می‌کند که گروه‌های متمرکز بر پایه خالص برای کشف مفاهیم پیچیده موثر نیستند . لازم به ذکر است که در گروه‌های متمرکز غیرهمزمان در متن , شرکت کنندگان تمایل به صحبت آزادانه بیشتری دارند , زیرا نمی‌توانند دیگران را ببینند . به طور خاص , پاسخ‌های ممکن است هدف بیشتری باشند , همانطور که شرکت کنندگان تمایل دارند مستقیما ً به نقطه برسند و وقتی که به صورت چهره‌به‌چهره نباشند , پاسخ دهند , چرا که پاسخ‌ها به جای صحبت کردن, تایپ می‌شوند
\end{frame}



\end{persian}
\end{document}