\documentclass[a4 paper,12pt]{article}\usepackage{xepersian}
\settextfont{Yas}
\setdigitfont{Yas}
\title{پروژه حروف چینی با نرم افزارلاتک}
\author{شهریار شمسی}
\date{\today}
\begin{document}
\maketitle



\noindent
نام : شهریار  نام خانوادگی : شمسی          شماره دانشجویی : 963930790 \\ دانشچوی رشته مهندسی کامپیوتر  \\ درس:  روش پژوهش و ارائه  \\ استاد محترم : جناب آقای دکتر رضوی \\

\noindent
 صفحات 103 تا 105  کتاب :                                                                                                                                                                                                                                                                                                                                                          \lr{E-search Methodes strategies}\

\vspace{0.1cm}
\vspace{0.1cm}

\begin{latin}
  \vspace{0.1cm}

\vspace{0.1cm}     
\noindent
others. While  the focus group interview was originally inspired in the  1950s as a way of obtaining consumer  product  preferences, this technique  eventually  found  its way into educational  and social science research  methodology.  It has been shown to have the capacity to garner rich and credible qualitative data. Further, according to Patton (1990), focus group  interviews provide quality controls on data collection, as partici• pants  tend to question  and eliminate false or extreme views. The result is a tendency to focus on the most important topics and issues and to assess the extent to which a rel• atively consistent, shared view exists among participants-as well as identifying  incon• sistent views.
\vspace{0.1cm}
This method  is called a focus group interview because it is focused in two ways. First, the group participants  are similar in some way (e.g., they have similar experiences of the topic being investigated).  Second,  the purpose  is to gather data about a single topic  (or a narrow  range  of topics).  They  are most often guided  by open-ended dis• cussion questions  proposed by the researcher,  with an emphasis on gaining insights through group opinions rather  than on specific facts. This format is a convenient way to  accumulate the individual  knowledge  of the members and to inspire  insights and solutions  that are difficult to achieve with other interview methods. A distinct advan• tage of focus groups is that they allow respondents to react to and build on responses. The result can be a synergistic and dynamic effect on group behavior, often resulting in data or ideas that might not have been collected in individual interviews (Stewart  Shamdasani, 1998). Moreover, because focus groups tend to provide checks and bal• ances among group members to eliminate false or extreme views, it is fairly easy for the researcher to assess the  extent  of consistent and shared views (Patton,  1990). Given these advantages, according to Glesne and Peskin (1992), interviewing  a group of peo• ple on a focused topic  can be a powerful way to collect data. However, it should be stressed that focus groups do not represent feedback from a randomly  selected popu• lation, but from purposely  selected individuals. As such, the results from focus group interviews should not be generalized to other, larger populations.
\vspace{0.1cm}
Finally,  Net-based focus  groups  are usually  selected  over face-to-face  focus groups  because  of the need  to involve individuals  from several different geographic areas. The travel time  and expense of bringing  geographically dispersed individuals together  is  often   prohibitive.  The  Net  provides   an  environment  whereby  the researcher can conduct  a focus group cost-effectively

\vspace{0.1cm}
THE DIFFERENT  KINDS OF NET-BASED  FOCUS GROUPS
\vspace{0.1cm}
\vspace{0.1cm}
\noindent
Net-based focus groups  can be conducted on  the Internet either  synchronously or asynchronously and with text-based  software and/or audio and video software. Table
8.1  provides   examples  of various  kinds  of Net-based  focus  groups   classified  by asynchronous, synchronous, text-based,  and non-text-based distinctions. As this table illustrates, there are four kinds ofNet-based focus groups: synchronous and text-based; synchronous and audio- and/or video-based; asynchronous and text-based; asynchro• nous and audio- and/or video-based. 


\vspace{0.1cm}


  

\vspace{0.1cm}

\vspace{0.1cm}


\vspace{0.1cm}

\vspace{0.1cm}

TABLE 8.1 Samples of Net-Based Systems to Support Focus  Groups\\
\begin{tabular}{cccc}\hline
                      &                             &syncronaise      & asyncronaise\\\hline
TEXT- BASED &                             & NET- METTING & Majordemo    \\\hline 
                      &                             &     ICQ              & FirstClass       \\\hline
                      &                             & FirstClass         & WebCT            \\\hline
                      &                             &   WebCT           & Email Groups    \\\hline 
AUDIO - AND/OR VIDEO BASED   &   Central           & Central             \\\hline
                      &                             &   Latitude          &   Latitude            \\\hline
                      &                             &  NET-METTING  &                            \\\hline



\end{tabular}

\vspace{0.1cm}
\vspace{0.1cm}
\vspace{0.1cm}


The  combination of media  used in the  focus group  process  makes it difficult  to generalize  about the  characteristics of all focus  groups.  Nevertheless, what uniquely differentiates a focus  group from an interview  is the  capacity of rarticipants  to share and build  on  the  comments and  concerns of other participants.Prior to the  Internet, there was no such  thing as an asynchronous focus group; although one  could conceive of mail-based focus  groups, researchers did  not  actively  use  the  technique. The pre­ dominant forms of asynchronous communication on the  Internet have been text-based email  and computer conferencing. Currently we are seeing a rapidly evolving selection of  more media-rich forms of asynchronous communication to  conduet Net-based focus  groups  (e.g.,  www.wimba.com), On the  synchronous side,  text-based chats are the  most common and  accessible  way to conduct real-time focus groups. This mode of interaction uses software such as ICQ,  NetMeeting, or one of the numerous Java-based Web chat  software  programs to share  the  comments of participants as they type. Text can be enhanced by viewing objects, sharing  applications, or sharing  a common space through text-based virtual reality (VR) systems such as MOO or  MUD (for frequently asked  questions about  MUDs and  MOO see: http://www.faqs.org/faqs/games/mud­ faq/part/) or  through two-  or  three-dimensional VR environments such  as  Palace (e.g.,  www.thepalace.com) or  virtual   worlds  (e.g., www.worlds.net).Finally, focus groups  can  be conducted  using  audio  or  video conferencing. Until   recently, the required  software,  end-user  hardware,  and  bandwidth  have prevented  use of these richer and more natural  forms  of communication on  the  Net.\\
 However,  the  develop­ ment  of multisite  audio  and  video  conferencing systems (see  www.microsoft.com/ windows/netmeeting/ and www.eentra.com) and  the  availability of  high-speed con­ nections at home and  in the workplace promise  increased use of media-rich, synchro­ nous  forms of Net-based  focus  groups.
Currently, most  Net-based  focus groups  are conducted using  text-based  asyn­ chronous or synchronous software.  Accordingly, this  chapter focuses  on  Net-based textual  focus groups. As broadband services proliferate and become more  widely avail­ able and affordable, we will likely see  an  increase  in synchronous and  asynchronous Net-based video and/or audio  focus groups. As these multimedia services are added to Net-based focus groups, they will tend to be more like face-to-face focus groups. Con­ sequently, conducting face-to-face focus groups will  become increasingly relevant and useful to e-researchers who use video- and audio-conferencing focus groups.

\noindent
  ADVANTAGES AND  DISADVANTAGES OF FACE-TO-FACE VERSUS  NET-BASED  FOCUS GROUPS\\
\vspace{0.1cm}

\noindent
Only recently have researchers been able to use the  Internet to conduct educationally related focus groups. Net-based focus groups offer both  speed and reduced cost  (Van Nuys,  1999) While Net-based focus groups are currently in the  exploration and  devel­ opment  stage, they appear to be especially effective at removing certain barriers that many  researchers experience when conducting face-to-face focus groups. In particular, they can  reduce or eliminate  participation and  cost  barriers. For example, if the  e­-researcher and the  participants are  geographically dispered,  Net-based  focus groups allow  them   to participate from their homes and/or offices, thus  travel   expenses are eliminated. Van Nuys's cost analysis indicates that, in addition to travel savings, there
is also  about a 20 percent cost savings in conducting the  focus group, compared to face­ to-face  focus groups. For example, such costs  as  food,  beverages, and room   rental would  not be incurred in a Net-based focus group. In addition to this  benefit, online discussions can  be automatically archived, eliminating the  transcription process and
transcriber interpretation error.
\vspace{0.1cm}


Finally, Net-based focus groups may also  reduce  power struggles that often occur in face-to-face focus groups asa result of conflicting opinions when there are perceived status differences among participants (Patton, 1990).  However, it  is not  necessary  for Net-based focus group participants to reveal  their real  identities to other members of the group. Given this ability  to provide an alias for each participant (depending on  the medium) and that Net-based focus groups have  the  ability  to join  geographically dis­ persed participants (thus reducing the  likelihood of participants knowing each  other), power struggles and confidentiality problems can  be reduced if not  eliminated.
\vspace{0.1cm}

\vspace{0.1cm}
Not with standing  these advantages, early   explorations with   Net-based focus
groups have met with  mixed results with  respect to the  quality of the  data collected. Van  Nuys (199) has  observed, for example, that  a drawback  of text-based asynchro­ nous focus groups is that  quite often there is less  depth in the participants' responses as  well as a loss of paralinguistic cues  (e.g., facial  expression, body posture, gesture, physical distance from  the  interlocutor, intonation pattern, and  volume). Paralinguis­ tie cues,  in particular, are considered to be a very  valuable source of data  in face-to-face
focus groups-in  addition to what is said.  Furthermore, Van Nuys notes that Net­
based focus groups tend not to be effective for  exploring complex concepts. Alterna­
tively, Van  Nuys has  also observed that in  text-based asynchronous focus groups, participants tend to speak more freely,  since they cannot see others. In particular, the responses may  be more objective, as participants tend to get  straight to the  point and not  to  "beat  around the  bush" when  they   are  not face-to-face, since responses are typed,  rather than spoken.



\end{latin}
\noindent
این گزارش و جدول پیوست آن همه   توسط نرم افزار لاتک تهیه شده است.
\end{document}



\end{document}

